\documentclass[11pt,a4paper,titlepage]{article}

\usepackage[top=2.5cm,left=1.5cm,right=1.5cm,bottom=2cm,headheight=13.6cm]{geometry}
\usepackage{color}
\usepackage{listings}
\usepackage[font=small,labelfont=bf]{caption} % figure caption
\usepackage{subcaption}
\usepackage{float}
\usepackage{titling}
\usepackage{amsmath} % cases
\usepackage{multicol}
\usepackage{xltxtra}
\usepackage{siunitx}
\usepackage{csvsimple}
\usepackage[nottoc,numbib]{tocbibind}
\usepackage[normalem]{ulem}
\usepackage[table,xcdraw]{xcolor}
\usepackage{fontspec} % set truetype font
\usepackage{pgfkeys}
\usepackage{soul} % fancy underlining
\usepackage{cite} % for range e.g [1-3]

%unicode fixer
%\usepackage[Latin,Greek,Phonetics]{ucharclasses} % please make unicode work. Please.
%\newfontfamily\substitutefont{Iosevka} % has most glyphs 
%\setTransitionsFor{IPAExtensions}{\begingroup\substitutefont}{\endgroup}

\setmonofont[Scale=MatchLowercase]{Iosevka}

% restore sensible table spacings
\renewcommand{\arraystretch}{1.5}

\definecolor{mygreen}{RGB}{0,127,0}
\definecolor{mygray}{RGB}{100,100,100}
\definecolor{mymauve}{RGB}{100,32,255}
\definecolor{lgray}{RGB}{230,230,230}
\settocbibname{Bibliography}
\lstset{ %
  frame=none,
  backgroundcolor=\color{white},   % choose the background color; you must add \usepackage{color} or \usepackage{xcolor}
  basicstyle=\footnotesize\ttfamily,        % the size of the fonts that are used for the code
  breakatwhitespace=false,         % sets if automatic breaks should only happen at whitespace
  breaklines=true,                 % sets automatic line breaking
  captionpos=t,                    % sets the caption-position to bottom
  commentstyle=\color{mygreen},    % comment style
  deletekeywords={...},            % if you want to delete keywords from the given language
  escapeinside={\%*}{*)},          % if you want to add LaTeX within your code
  extendedchars=true,              % lets you use non-ASCII characters; for 8-bits encodings only, does not work with UTF-8
%  frame=single,                    % adds a frame around the code
  keepspaces=true,                 % keeps spaces in text, useful for keeping indentation of code (possibly needs columns=flexible)
  keywordstyle=\color{blue},       % keyword style
  language=,                 % the language of the code
  morekeywords={*,...},            % if you want to add more keywords to the set
  numbers=left,                    % where to put the line-numbers; possible values are (none, left, right)
  numbersep=5pt,                   % how far the line-numbers are from the code
  numberstyle=\tiny\color{mygray}, % the style that is used for the line-numbers
  rulecolor=\color{black},         % if not set, the frame-color may be changed on line-breaks within not-black text (e.g. comments (green here))
  showspaces=false,                % show spaces everywhere adding particular underscores; it overrides 'showstringspaces'
  showstringspaces=false,          % underline spaces within strings only
  showtabs=false,                  % show tabs within strings adding particular underscores
  stepnumber=1,                    % the step between two line-numbers. If it's 1, each line will be numbered
  stringstyle=\color{mymauve},     % string literal style
  tabsize=4,                       % sets default tabsize to 2 spaces
  aboveskip=3mm,
  belowskip=3mm,
}

\usepackage{fancyhdr}
\usepackage{datetime}
\usepackage{moresize}
\usepackage{titlesec}
%  Clickable TOC and refs
\usepackage{hyperref}
\hypersetup{
    colorlinks,
    citecolor=black,
    filecolor=black,
    linkcolor=black,
    urlcolor=black
}
% Fix fonts in TOC
\usepackage{tocloft}

\pagestyle{fancy}
\rhead{\lightfont Y3761870}
\renewcommand{\headrulewidth}{2pt}% 2pt header rule
\renewcommand{\headrule}{\hbox to\headwidth{%
 \color{lgray}\leaders\hrule height \headrulewidth\hfill}}

% table of contents depth
\setcounter{tocdepth}{2}
  
\setsansfont{DINPro-Bold.otf}
\newfontfamily\boldfont{DINPro-Bold.otf}
\newfontfamily\mediumfont{DINPro-Medium.otf}
\newfontfamily\regularfont{DINPro-Regular.otf}
\newfontfamily\lightfont{DINPro-Light.otf}
%\newfontfamily\ttfamily{iosevka-custom-regular.ttf} % too big
% Set formats for each heading level
\titleformat*{\section}{\Large\bfseries\sffamily}
\titleformat*{\subsection}{\large\bfseries\mediumfont}
\titleformat*{\subsubsection}{\itshape\mediumfont}

\renewcommand{\contentsname}{C O N T E N T S}
\renewcommand{\cfttoctitlefont}{\small\regularfont\MakeUppercase}
\renewcommand{\cftsecfont}{\regularfont}
\renewcommand{\cftsecpagefont}{\mediumfont}
\renewcommand{\cftsubsecfont}{\lightfont}
\renewcommand{\cftsubsecpagefont}{\regularfont}
% Adjust spacing in TOC
\setlength\cftbeforesubsecskip{5pt}
% remove annoying space before maths
\setlength{\abovedisplayskip}{0pt}


% redefine footer
\fancyfoot[C]{\regularfont\fontsize{11pt}{11pt}\selectfont\thepage} % except the center


\newdateformat{monthyeardate}{%
  \monthname[\THEMONTH] \THEYEAR}

\newcommand{\rulebreak}{%
	\par%
	\vspace{0.9cm}%
    \noindent\color{lgray}\rule{4cm}{2pt}%
    \color{black}%
    \vspace{1.2cm}%
    \par%
}

\newcommand{\coverpage}[1]{%
	\pagenumbering{gobble}%
	\thispagestyle{empty}%
	\lhead{\lightfont\textsc{#1}}%
    \title{Voice Production \& Synthesis}%
    \author{Y3761870}%
    \newgeometry{left=5cm,bottom=2cm,right=5cm,top=2cm}%
	\begin{center}\hspace{0pt}\vfill%
    \uppercase{\lightfont Department of Electronic Engineering\\
    The University of York}
	\rulebreak%
    {\Large\textbf{\sffamily Voice Production \& Synthesis}}
    
    \vspace{0.5cm}
    {\HUGE\textbf{\textit{\sffamily #1}}\par}
    \vspace{1cm}
    
	\theauthor%
	\par%
	\vspace{0.9cm}%
    \noindent\color{lgray}\rule{4cm}{2pt}\color{black}%
    \vspace{0.45cm}
    \tableofcontents%
	\rulebreak%
    \monthyeardate\today\par
    \hspace{0pt}
	\end{center}%
    \vfill
    \hspace{0pt}
	\pagebreak%
    \restoregeometry%
    \pagenumbering{arabic}%
}

\newcommand{\filename}[1]{%
	\texttt{#1}%
}

\newcommand{\ipa}[1]{%
	\texttt{/#1/}%
}

\renewcommand{\~}{\char`~}

\newcommand*{\obj}[1]{\texttt{[#1]}}
\newcommand*{\msg}[1]{\texttt{[#1(}}


\newcommand*\paths[1]{\lstset{inputpath=#1}\graphicspath{#1}}

